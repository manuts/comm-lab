\documentclass[t]{beamer}
\usepackage{mathtools}
\usepackage{tikz}
\usepackage{pgfplots}
\usetikzlibrary{arrows,backgrounds,shapes,matrix,positioning,fit}
\newcommand{\argmax}{\operatornamewithlimits{argmax}}
\newcommand{\argmin}{\operatornamewithlimits{argmin}}
\newcommand{\wt}{\operatornamewithlimits{wt}}
\newcommand{\cov}{\operatornamewithlimits{cov}}
\newcommand{\var}{\operatornamewithlimits{var}}

\mode<presentation>
{
  \usetheme{Singapore}
  %\useoutertheme{infolines} % Showing only current section in navigation
  \setbeamertemplate{headline}{}  % Empty headline
  \setbeamertemplate{footline}[frame number]  % Getting rid of footer items except slide number
  \setbeamercovered{invisible}
  \beamertemplatenavigationsymbolsempty % Getting rid of navigation bullets at the bottom
}
\usepackage[english]{babel}
\usepackage[latin1]{inputenc}
\usepackage{times}
\usepackage[T1]{fontenc}

\title[EE 703 DMT]{Random Processes}
\author[Saravanan V]
{
  Saravanan Vijayakumaran\\
  \href{mailto:sarva@ee.iitb.ac.in}{sarva@ee.iitb.ac.in}
}
\institute[IIT Bombay]
{
  Department of Electrical Engineering\\
  Indian Institute of Technology Bombay
}
\date{August 12, 2013}

\AtBeginSection[]%
{%
\begin{frame}[plain]%
  \topskip0pt
  \vspace*{\fill}
    \begin{center}%
      \usebeamerfont{section title}\insertsection%
    \end{center}%
  \vspace*{\fill}
\end{frame}%
}

\begin{document}

\begin{frame}
  \titlepage
\end{frame}

%% Frame %%
\begin{frame}{Random Process}
  \footnotesize
  \pause
  \begin{definition}[]
  An indexed collection of random variables $\{X(t):t \in \mathcal{T}\}$.
  \end{definition}
  \begin{description}
    \pause
    \item[Discrete-time Random Process] $\mathcal{T} = \pause \mathbb{Z} \text{ or } \mathbb{N}$
    \pause
    \item[Continuous-time Random Process] $\mathcal{T} = \pause \mathbb{R}$
  \end{description}
  \pause
  \begin{block}{Statistics}
  Mean function
  \pause
    \begin{equation*}
      m_X(t) = \pause E[X(t)]
    \end{equation*}
  \pause
  Autocorrelation function
  \pause
    \begin{equation*}
      R_X(t_1,t_2) = \pause E[X(t_1)X^*(t_2)]
    \end{equation*}
  \pause
  Autocovariance function
  \pause
    \begin{equation*}
      C_X(t_1,t_2) = \pause E\left[\left(X(t_1)-m_X(t_1)\right)\left(X(t_2)-m_X(t_2)\right)^*\right]
    \end{equation*}
  \end{block}
  \normalsize
\end{frame}

%% Frame %%
\begin{frame}{Stationary Random Process}
  \footnotesize
  \pause
  \begin{definition}[]
    A random process which is statistically indistinguishable from a delayed version of itself.
  \end{definition}
  \pause
  \begin{block}{Properties}
  \begin{itemize}
    \item For any $n \in \mathbb{N}$, $(t_1,\ldots,t_n) \in \mathbb{R}^n$ and $\tau \in \mathbb{R}$, $\left(X(t_1),\ldots,X(t_n)\right)$ has the same joint distribution as $\left(X(t_1-\tau),\ldots,X(t_n-\tau)\right)$.
     \item \pause For $n = 1$, we have
     \begin{equation*}
       F_{X(t)}(x) = F_{X(t+\tau)}(x)
     \end{equation*}
      for all $t$ and $\tau$. \pause The first order distribution is independent of time.
    \item \pause $m_X(t) = \pause m_X(0)$
    \item \pause For $n = 2$ and $\tau = -t_1$, we have
       \begin{equation*}
         F_{X(t_1), X(t_2)}(x_1, x_2) = F_{X(0),X(t_2-t_1)}(x_1, x_2)
       \end{equation*}
    for all $t_1$ and $t_2$. \pause The second order distribution depends only on $t_2 - t_1$.
    \item \pause $R_X(t_1,t_2) = \pause R_X(t_1-\tau,t_2-\tau) = \pause R_X(t_1-t_2,0)$
  \end{itemize}
  \end{block}
  \normalsize
\end{frame}

%% Frame %%
\begin{frame}{Wide Sense Stationary Random Process}
  \footnotesize
  \pause
  \begin{definition}[]
    A random process is WSS if
    \begin{eqnarray*}
      m_X(t) & = & m_X(0) \ \ \ \ \ \text{ for all $t$ and }\\
      \pause
      R_X(t_1,t_2) & = & R_X(t_1-t_2,0) \ \ \ \ \ \text{for all $t_1,t_2$.} \\
    \end{eqnarray*}
    \pause
    Autocorrelation function is expressed as a function of $\tau = t_1-t_2$ as $R_X(\tau)$.
  \end{definition}
  \pause
  \begin{definition}[Power Spectral Density of a WSS Process]
    \pause
    The Fourier transform of the autocorrelation function.
    \pause
    \begin{equation*}
      S_X(f) = \mathcal{F}\left(R_X(\tau)\right)
    \end{equation*}
  \end{definition}
  \normalsize
\end{frame}

%% Frame %%
\begin{frame}{Energy Spectral Density}
  \footnotesize
  \pause
  \begin{definition}[]
    For a signal $s(t)$, the energy spectral density is defined as
    \begin{equation*}
      E_s(f) = \lvert S(f) \rvert^2.
    \end{equation*}
  \end{definition}
  \pause
  \begin{block}{Motivation}
    \pause
    Pass $s(t)$ through an ideal narrowband filter with response
    \begin{equation*}
    H_{f_0}(f) = \left\{
          \begin{array}{rr}
          1, & \text{if } f_0 - \frac{\Delta f}{2} < f < f_0 + \frac{\Delta f}{2} \\
          0, & \text{otherwise} 
          \end{array}
        \right.
    \end{equation*}
    \pause
    Output is $Y(f) = S(f)H_{f_0}(f)$. \pause  Energy in output is given by
    \begin{equation*}
      \int_{-\infty}^{\infty} \lvert Y(f) \rvert^2 \ df = \pause \int_{f_0-\frac{\Delta f}{2}}^{f_0+\frac{\Delta f}{2}} \lvert S(f) \rvert^2 \ df  \approx  \pause\lvert S(f_0) \rvert^2 \Delta f 
    \end{equation*}
  \end{block}
  \normalsize
\end{frame}

%% Frame %%
\begin{frame}{Power Spectral Density}
  \footnotesize
  \pause
  \begin{block}{Motivation}
    PSD characterizes spectral content of random signals which have infinite energy but finite power 
  \end{block}
    \pause
    \begin{example}[Finite-power infinite-energy signal] 
    \pause 
    Binary PAM signal
      \begin{equation*}
      x(t) = \sum_{n=-\infty}^{\infty} b_n p(t-nT)
      \end{equation*}
    \end{example}
  \normalsize
\end{frame}

%% Frame %%
\begin{frame}{Power Spectral Density of a Realization}
  \footnotesize
    Time windowed realizations have finite energy
    \begin{eqnarray*}
      x_{T_o}(t) & = & x(t)I_{[-\frac{T_o}{2},\frac{T_o}{2}]}(t) \\
      \pause
      S_{T_o}(f) & = & \mathcal{F}(x_{T_o}(t)) \\
      \pause
      \hat{S}_x(f) & = & \frac{\lvert S_{T_o}(f) \rvert^2}{T_o} \ \ \ \ \text{ (PSD Estimate)}
    \end{eqnarray*}
  \pause
  \begin{definition}[PSD of a realization]
    \begin{equation*}
    \bar{S}_x(f) = \lim_{T_o \rightarrow \infty} \frac{\lvert S_{T_o}(f) \rvert^2}{T_o}
    \end{equation*}
  \end{definition}
  \normalsize
\end{frame}

%% Frame %%
\begin{frame}{Autocorrelation Function of a Realization}
  \footnotesize
  \pause
  \begin{block}{Motivation}
    \begin{eqnarray*}
     \hat{S}_x(f)  =  \frac{\lvert S_{T_o}(f) \rvert^2}{T_o} & \xrightleftharpoons{} & \pause\frac{1}{T_o} \int_{-\infty}^{\infty} x_{T_o}(u)x_{T_o}^*(u-\tau) \ du \\
     \pause
     &  & =  \frac{1}{T_o} \int_{-\frac{T_o}{2}}^{\frac{T_o}{2}} x_{T_o}(u)x_{T_o}^*(u-\tau) \ du \\
     \pause
     &  & = \hat{R}_x(\tau) \ \ \ \ \ \ \text{(Autocorrelation Estimate)}
    \end{eqnarray*}
  \end{block}
  \pause
  \begin{definition}[Autocorrelation function of a realization]
    \begin{equation*}
      \bar{R}_x(\tau) = \lim_{T_o \rightarrow \infty} \frac{1}{T_o} \int_{-\frac{T_o}{2}}^{\frac{T_o}{2}} x_{T_o}(u)x_{T_o}^*(u-\tau) \ du 
    \end{equation*}
  \end{definition}
  \normalsize
\end{frame}

%% Frame %%
\begin{frame}{The Two Definitions of Power Spectral Density}
  \footnotesize
  \begin{definition}[PSD of a WSS Process]
    \pause
    \begin{equation*}
      S_X(f) = \mathcal{F}\left(R_X(\tau)\right)
    \end{equation*}
    where $R_X(\tau) = E\left[ X(t)X^*(t-\tau)\right]$.
  \end{definition}
  \pause
  \begin{definition}[PSD of a realization]
    \begin{equation*}
    \bar{S}_x(f) = \mathcal{F}\left(\bar{R}_x(\tau)\right)
    \end{equation*}
    where
    \begin{equation*}
      \bar{R}_x(\tau) = \lim_{T_o \rightarrow \infty} \frac{1}{T_o} \int_{-\frac{T_o}{2}}^{\frac{T_o}{2}} x_{T_o}(u)x_{T_o}^*(u-\tau) \ du 
    \end{equation*}
  \end{definition}
  \pause
  \begin{center}
    Both are equal for ergodic processes
  \end{center}
  \normalsize
\end{frame}

%% Frame %%
\begin{frame}{Ergodic Process}
  \footnotesize
  \pause
  \begin{definition}[]
    A stationary random process is ergodic if time averages equal ensemble averages.
  \end{definition}
  \pause
  \begin{itemize}
    \item Ergodic in mean
      \begin{equation*}
        \lim_{T\rightarrow\infty} \frac{1}{T} \int_{-\frac{T}{2}}^{\frac{T}{2}} x(t) \ dt = E[X(t)]
      \end{equation*}
    \pause
    \item Ergodic in autocorrelation
      \begin{equation*}
        \lim_{T\rightarrow\infty} \frac{1}{T} \int_{-\frac{T}{2}}^{\frac{T}{2}} x(t)x^*(t-\tau) \ dt = R_X(\tau)
      \end{equation*}
  \end{itemize}
  \normalsize
\end{frame}

%% Frame %%
\begin{frame}{}
\vfill
\begin{center}
Thanks for your attention
\end{center}
\vfill
\end{frame}


\end{document}
