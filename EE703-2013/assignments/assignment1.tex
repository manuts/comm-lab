\documentclass[10pt]{report}
\usepackage{./EE703handout}
\usepackage{tikz}
\usepackage{pgfplots}
\usetikzlibrary{decorations.pathmorphing}
\usepackage{mathtools}
\usepackage{subfig}
\usepackage{hyperref}
\usepackage{enumitem}
\usepackage{verbatim}
\usetikzlibrary{arrows,backgrounds,shapes,matrix,positioning,fit}
\setlength{\textheight}{9.4in}     % 9.4in
\setlength{\textwidth}{6.3in}      % 6.3in
\setlength{\parindent}{0mm}
%\setlength{\parskip}{3mm}
\setlength{\oddsidemargin}{0.0in}  % 0.0in
\setlength{\topmargin}{-0.0in}  % 0.0in
\setlength{\headheight}{0.0in}  % 0.0in


\begin{document}
\handout{}{Due Date: August 8, 2013}{Assignment 1}
\begin{enumerate}
  \item Suppose we define the complex envelope of a passband signal $s_p(t)$ centered at $\pm f_c$ as  
    \begin{eqnarray*}
      S(f) = 2S_p(f - f_c)u(- f + f_c)
    \end{eqnarray*}
  where $S_p(f)$ is the Fourier transform of $s_p(t)$. Derive the following with explanations for each step.
    \begin{enumerate}
      \item $s_p(t)$ in terms of $s(t)$
      \item $s_p(t)$ in terms of $s_c(t)$ and $s_s(t)$ (the in-phase and quadrature components of $s(t)$)
      \item $s(t)$ in terms of $s_p(t)$
      \item $S_p(f)$ in terms of $S(f)$
      \item The relationship between $\lVert s \rVert^2$ and $\lVert s_p \rVert^2$.
    \end{enumerate}
  \item For a real baseband signal $s(t)$, the corresponding single side band (SSB) modulated signal is given by
    \begin{eqnarray*}
      s_{ssb}(t) = s(t) \cdot \cos (2\pi f_c t) - \hat{s}(t) \cdot \sin (2\pi f_c t)
    \end{eqnarray*}
  where $\hat{s}(t)$ is the Hilbert transform of $s(t)$.
    We saw in class that the frequency spectrum of $s_{ssb}(t)$ only has the upper sideband of $s(t)$. Suppose we want a SSB signal which has only the lower sideband of $s(t)$. What is the time domain representation of such a signal in terms of $s(t)$ and $\hat{s}(t)$?
\end{enumerate}
\end{document}
