\documentclass[10pt]{report}
\usepackage{./EE703handout}
\usepackage{tikz}
\usepackage{pgfplots}
\usetikzlibrary{decorations.pathmorphing}
\usepackage{mathtools}
\usepackage{subfig}
\usepackage{hyperref}
\usepackage{enumitem}
\usepackage{verbatim}
\usepackage{amssymb}
\usetikzlibrary{arrows,backgrounds,shapes,matrix,positioning,fit}
\setlength{\textheight}{9.4in}     % 9.4in
\setlength{\textwidth}{6.3in}      % 6.3in
\setlength{\parindent}{0mm}
%\setlength{\parskip}{3mm}
\setlength{\oddsidemargin}{0.0in}  % 0.0in
\setlength{\topmargin}{-0.5in}  % 0.0in
\setlength{\headheight}{0.0in}  % 0.0in

\renewcommand\Re{\operatorname{Re}}
\renewcommand\Im{\operatorname{Im}}

\begin{document}
\handout{}{Due Date: October 24, 2013}{Assignment 5}
\begin{enumerate}
  \item For the 16-QAM constellation shown below calculate $E_b$ in terms of $A$. Assume that the transmitted symbol is corrupted by adding $N\sim \mathcal{CN}(0,N_0)$. If all the constellation points are equally likely to be transmitted, calculate the following in terms of $E_b$ and $N_0$.
  \begin{itemize}
    \item The exact error probability of the optimal decision rule.
    \item The union bound on the exact error probability.
    \item The intelligent union bound on the exact error probability.
    \item The nearest neighbor approximation of the exact error probability.
  \end{itemize}
  \begin{figure}[h]
    \centering
      \begin{tikzpicture}[scale=1.0,transform shape]
        \begin{axis}[
                     xmax=5,
                     xmin=-5,
                     ymax=5,
                     ymin=-5,
                     axis lines = middle,
                     xtick = {-3, -1, 1, 3},
                     xticklabels = {$-3A$, $-A$, $A$, $3A$},
                     x tick label style = {font = \tiny},
                     ytick = {-3, -1, 1, 3},
                     yticklabels = {$-3A$, $-A$, $A$, $3A$},
                     y tick label style = {font = \tiny},
                     y axis line style={-},
                     x axis line style={-},
                    ]
          \addplot+[only marks, 
                    mark options={draw=black,fill=black},
                    ] 
                    coordinates {
                      (3,1)
                      (1,3)
                      (-1,3)
                      (-3,1)
                      (-3,-1)
                      (-1,-3)
                      (1,-3)
                      (3,-1)
                      (1,1)
                      (-1,1)
                      (1,-1)
                      (-1,-1)
                      (3,3)
                      (-3,3)
                      (3,-3)
                      (-3,-3)
                    };
        \end{axis}
      \end{tikzpicture}
  \end{figure}
  \item In the table given below, show that the modulation schemes in the first column have the power efficiencies in the second column.
    \begin{table}[h]
      \centering
      \begin{tabular}{ll} 
        Modulation Scheme & $\eta_p$ \\ \hline
        Orthogonal signaling & 2 \\
        Antipodal signaling & 4 \\
        BPSK & 4 \\
        QPSK & 4 \\
        16-QAM & 1.6 \\
      \end{tabular}
    \end{table}

\end{enumerate}
\end{document}
