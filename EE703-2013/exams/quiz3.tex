\documentclass[10pt]{report}
\usepackage{./EE703handout}
\usepackage{tikz}
\usepackage{pgfplots}
\usetikzlibrary{decorations.pathmorphing}
\usepackage{mathtools}
\usepackage{subfig}
\usepackage{hyperref}
\usepackage{enumitem}
\usepackage{verbatim}
\usetikzlibrary{arrows,backgrounds,shapes,matrix,positioning,fit}
\setlength{\textheight}{9.4in}     % 9.4in
\setlength{\textwidth}{6.3in}      % 6.3in
\setlength{\parindent}{0mm}
%\setlength{\parskip}{3mm}
\setlength{\oddsidemargin}{0.0in}  % 0.0in
\setlength{\topmargin}{-1.0in}  % 0.0in
\setlength{\headheight}{0.0in}  % 0.0in
\pgfplotsset{compat=1.5.1}

\begin{document}
\handout{}{Date: November 11, 2013}{Quiz 3: \textbf{12 points}}
\begin{enumerate}
  \item Suppose that the 8-PSK constellation shown below is used over a complex AWGN channel with PSD $N_0$.
    Assume that all the eight signals are equally likely to be transmitted.
  \begin{figure}[h]
    \centering
      \begin{tikzpicture}[scale=1.0,transform shape]
        \begin{axis}[
                     axis equal,
                     xmax=1.7,
                     xmin=-1.7,
                     ymax=1.7,
                     ymin=-1.7,
                     axis lines = middle,
                     ytick={1.65},
                     yticklabels = {$$},
                     xtick={1.7},
                     xticklabels = {$$},
                     xticklabel shift = -20pt,
                     nodes near coords,
                    ]
          \addplot+[only marks, 
                    mark options={draw=black,fill=black},
                    point meta=explicit symbolic
                    ] 
                    coordinates {
                      (1.414,0)[$s_1$]
                      (1,1)[$s_2$]
                      (0,1.414)[$s_3$]
                      (-1,1)[$s_4$]
                      (-1.414,0)[$s_5$]
                      (-1,-1)[$s_6$]
                      (0,-1.414)[$s_7$]
                      (1,-1)[$s_8$]
                    };
          \draw[->] (axis cs:0,0) -- (axis cs:0.96,0.96) node [midway, sloped, above] {$R$};
        \end{axis}
      \end{tikzpicture}
  \end{figure}
    \begin{enumerate}
      \item Express $E_b$ as a function of $R$.
      \item Derive the power efficiency of this modulation scheme. \textit{Hint:} $\sin \frac{\pi}{8} \approx \frac{3}{8}$.
      \item Calculate \textbf{any two} of the following three quantities as a function of $E_b$ and $N_0$.
        \begin{enumerate}
          \item The exact symbol error probability of the ML receiver.
          \item The intelligent union bound on the symbol error probability of the ML receiver.
          \item The nearest neighbor approximation to the symbol error probability of the ML receiver.
        \end{enumerate}
    \end{enumerate}
  \item Suppose observations $Y_i$, $i=1,2,\ldots,N$ are Poisson distributed with parameter $\lambda$. Assume that the $Y_i$'s are independent.
    \begin{enumerate}
      \item Derive the ML estimator for $\lambda$.
      \item Find the mean and variance of the ML estimate.
    \end{enumerate}
  Recall that a Poisson distributed random variable with parameter $\lambda$ has a probability mass function given by
    \begin{equation*}
      \Pr(Y = n) = \frac{e^{-\lambda}\lambda^n}{n!}, n = 0,1,2,\ldots
    \end{equation*}
  with mean and variance both equal to $\lambda$.
  \item The following set of four signals is used to send two bits over a baseband AWGN channel with PSD $\frac{N_0}{2}$. 
    \begin{equation*}
      s_1(t) = -2A p(t), s_2(t) = -A p(t), s_3(t) = A p(t), s_4(t) = 2A p(t)
    \end{equation*}
    where $p(t) = I_{[0,1]}(t)$ and 
    Assume that all the four signals are equally likely to be transmitted.
    \begin{enumerate}
      \item Derive the power efficiency of this modulation scheme.
      \item Specify a Gray code for mapping each symbol to 2 bits.
      \item Calculate the bit error probability of the ML receiver in terms of $E_b$ and $N_0$ assuming the Gray code in part (b).
    \end{enumerate}
\end{enumerate}
\end{document}
