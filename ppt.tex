\documentclass[t]{beamer}
\usepackage{mathtools}
\usepackage{tikz}
\usepackage{pgfplots}
\usepackage{ifthen}
\usepackage{calc}
\usepackage{datatool}
\usepackage{datapie}
\usetikzlibrary{arrows,backgrounds,shapes,matrix,positioning,fit}
\usetikzlibrary{calendar}
\newcommand{\argmax}{\operatornamewithlimits{argmax}}
\newcommand{\argmin}{\operatornamewithlimits{argmin}}
\newcommand{\wt}{\operatornamewithlimits{wt}}

\mode<presentation>
{
  \usetheme{Singapore}
  %\useoutertheme{infolines} % Showing only current section in navigation
  \setbeamertemplate{headline}{}  % Empty headline
  \setbeamertemplate{footline}[frame number]  % Getting rid of footer items except slide number
  \setbeamercovered{invisible}
  \beamertemplatenavigationsymbolsempty % Getting rid of navigation bullets at the bottom
}
\usepackage[english]{babel}
\usepackage[latin1]{inputenc}
\usepackage{times}
\usepackage[T1]{fontenc}
\usepackage{mathtools}
\usepackage{tikz}
\usepackage{pgfplots}
\usetikzlibrary{arrows,backgrounds,shapes,matrix,positioning,fit}
\usetikzlibrary{decorations.markings}
\usepackage{caption}
\usepackage{subcaption}

\title[IQ Modulator]{IQ Modulator Board}
\author[Manu]
{
  Manu T S\\
  \href{mailto:manu.ts@ee.iitb.ac.in}{manu.ts@ee.iitb.ac.in}
}
\institute[IIT Bombay]
{
  Department of Electrical Engineering\\
  Indian Institute of Technology Bombay
}
\date{\today}

\begin{document}

\begin{frame}
  \titlepage
\end{frame}

\begin{frame}{IQ Modulation}
  Any passband signal $s_{\text{p}}(t)$ can be written as
  \begin{equation}
      s_{\text{p}}(t) = \sqrt{2}s_{\text{i}}(t)\operatorname{cos}(2\pi f_ct) - \sqrt{2}s_{\text{q}}(t)\operatorname{sin}(2\pi f_ct)
  \end{equation}
  where,
  \begin{equation}
    s(t) = s_{\text{i}}(t) + \text{j}s_{\text{q}}(t)
  \end{equation}
\end{frame}

% \begin{frame}{Course Details}
%   \begin{description}
%     \item[Instructor] Saravanan Vijayakumaran
%     \item[Office] EE 122B (Opposite PC Lab)
%     \item[Schedule] Slot 1
%     \item[Location] EEG 002
%     \item[Webpage] \url{http://www.ee.iitb.ac.in/~sarva/EE703/Autumn2013.html}
%     \item[Forums] \url{https://piazza.com/iit_bombay/other/ee703}
%   \end{description}
% \end{frame}
% 
% %% Database for marks distribution
% \DTLnewdb{gradingpolicy}
% \DTLnewrow{gradingpolicy}
% \DTLnewdbentry{gradingpolicy}{Name}{Endsem}
% \DTLnewdbentry{gradingpolicy}{Points}{45}
% \DTLnewrow{gradingpolicy}
% \DTLnewdbentry{gradingpolicy}{Name}{Midsem}
% \DTLnewdbentry{gradingpolicy}{Points}{30}
% \DTLnewrow{gradingpolicy}
% \DTLnewdbentry{gradingpolicy}{Name}{Quizzes}
% \DTLnewdbentry{gradingpolicy}{Points}{15}
% \DTLnewrow{gradingpolicy}
% \DTLnewdbentry{gradingpolicy}{Name}{Assignments}
% \DTLnewdbentry{gradingpolicy}{Points}{10}
% %% End database
% 
% \begin{frame}{Grading Policy}
%   \begin{figure}
%   \centering
%   \setcounter{DTLpieroundvar}{0}
%   \DTLpiechart{variable=\points,outerlabel=\name,innerratio=0.7,outerratio=1.1,innerlabel={\DTLpiepercent\%}}
%               {gradingpolicy}
%               {\points=Points,\name=Name}
%   \end{figure}
% 
%   \begin{itemize}
%     \item Quizzes (Best two out of three)
%     \item Relative grading
%     \item For AU, score $\geq$ CC
%   \end{itemize}
% \end{frame}
% 
% \begin{frame}{Reference Books}
%   \begin{itemize}
%     
%     \item \textit{Fundamentals of Digital Communication}, Upamanyu Madhow, 2008
%     \item \textit{Digital Communications}, John G. Proakis and Masoud Salehi, 2008 (5th Edition)
%   \end{itemize}
% \end{frame}
% 
% 
% \begin{frame}{Digital Communication Systems}
%   \begin{figure}
%   \centering
%   \begin{tikzpicture}[scale=0.55,transform shape]
%   \tikzstyle{rectblock}=[rectangle, draw, inner sep=3mm]
%   \node[rectblock] (Source) {\begin{tabular}{c} Information\\
%                                                 Source
%                              \end{tabular}};
%   \node[rectblock, right=1.5cm of Source] (Source Encoder) {\begin{tabular}{c}Source\\
%                                                                              Encoder
%                                                            \end{tabular}};
%   \node[rectblock, right=1.5cm of Source Encoder] (Channel Encoder) {\begin{tabular}{c}Channel\\
%                                                                              Encoder
%                                                            \end{tabular}};
%   \node[rectblock, right=1.5cm of Channel Encoder] (Modulator) {Modulator};
%   \node[rectblock, below right=1.5cm of Modulator] (Channel) {Channel};
%   \node[rectblock, below left=1.5cm of Channel] (Demodulator) {Demodulator};
%   \node[rectblock, left=1.5cm of Demodulator] (Channel Decoder) {\begin{tabular}{c}Channel\\
%                                                                                Decoder
%                                                              \end{tabular}};
%   \node[rectblock, left=1.5cm of Channel Decoder] (Source Decoder) {\begin{tabular}{c}Source\\
%                                                                                Decoder
%                                                              \end{tabular}};
%   \node[rectblock, left=1.5cm of Source Decoder] (Destination) {\begin{tabular}{c} Information\\
%                                                                                    Destination
%                                                                 \end{tabular}};
%   \draw [->,very thick] (Source) -- (Source Encoder);
%   \draw [->,very thick] (Source Encoder) -- (Channel Encoder);
%   \draw [->,very thick] (Channel Encoder) -- (Modulator);
%   \draw [->,very thick] (Modulator) -| (Channel);
%   \draw [->,very thick] (Channel) |- (Demodulator);
%   \draw [->,very thick] (Demodulator) -- (Channel Decoder);
%   \draw [->,very thick] (Channel Decoder) -- (Source Decoder);
%   \draw [->,very thick] (Source Decoder) -- (Destination);
%   \tikzset{dotted/.style={draw=black!50!white, line width=1pt,
%                          dash pattern=on 1pt off 1pt,
%                          inner sep=4mm, rectangle, rounded corners}};
%   \pause
%   \node (EE708 dotted box) [dotted, fit = (Source Encoder) (Source Decoder)] {};
%   \node at (EE708 dotted box.north) [above, inner sep=3mm] {\textbf{EE 708}};
%   \pause
%   \node (EE605 dotted box) [dotted, fit = (Channel Encoder) (Channel Decoder)] {};
%   \node at (EE605 dotted box.north) [above, inner sep=3mm] {\textbf{EE 605}};
%   \pause
%   \node (EE703 dotted box) [dotted, fit = (Modulator) (Demodulator) (Channel)] {};
%   \node at (EE703 dotted box.north) [above, inner sep=3mm] {\alert{\textbf{EE 703}}};
%   \end{tikzpicture}
%   \label{fig:commsys}
%   \end{figure}
% 
%   \begin{description}
%     \item<2->[EE 708] Information Theory and Coding
%     \item<3->[EE 605] Error Correcting Codes
%     \item<4->[EE 703] Digital Message Transmission
%   \end{description}
% \end{frame}
% 
% \begin{frame}{Course Outline}
%   \begin{itemize}
%     \item Review of Prerequisites
%     \item Digital Modulation Schemes
%     \item Demodulation Schemes
%     \item Carrier and Timing Synchronization
%     \item Equalization
%   \end{itemize}
% \end{frame}
% 
% \begin{frame}{Lecture Distribution by Month}
%   \begin{figure}
%   \centering
%   \begin{tikzpicture}[scale=0.85,transform shape]
%     \begin{axis}[ybar stacked, 
%                  ylabel = Number of Lectures,
%                  grid = major,
%                  ymin = 0,
%                  ymax = 15,
%                  ytick = {0,1,...,15},
%                  xtick = {1,3,5,7,9},
%                  xticklabels = {July,August,September,October,November},
%                  %symbolic x coords={Jul,Aug,Sep,Oct,Nov},
%                  legend style={at={(0.5,-0.15)}, anchor = north, legend columns=-1},
%                  x post scale = 1.5
%                  ]
%       \addplot[fill=blue, bar width = 20] coordinates
%       {(1,6) (3, 12) (5, 3) (7,0) (9,0)};
%       %{(Jul,3) (Aug, 9) (Sep, 2) (Oct,0) (Nov,0)};
%       \addplot[fill=orange, bar width = 20] coordinates
%       {(1, 0) (3, 0) (5, 6) (7,14) (9, 4)};
%       %{(Jul, 0) (Aug, 0) (Sep, 3) (Oct,8) (Nov, 5)};
%     \legend{Before Midsem, After Midsem}
%     \end{axis}
%   \end{tikzpicture}
%   \end{figure}
% 
%   \begin{description}
%     \item[Before Midsem] 21 Lectures
%     \item[After Midsem] 24 Lectures
%     \pause
%     \item[Attendance] \alert{80\% required} (Can miss at most nine lectures)
%   \end{description}
% \end{frame}
% 
% \begin{frame}{}
% \vfill
% \begin{center}
% End of course outline
% \end{center}
% \vfill
% \end{frame}
\end{document}
