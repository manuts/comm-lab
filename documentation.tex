\documentclass[a4paper,10pt]{report}
\usepackage[utf8]{inputenc}
\usepackage{lipsum}
% For maths
\usepackage{amsmath}
% Set Margin
\usepackage[margin=1in]{geometry}
% For Figures
\usepackage{graphicx}
% For including source codes
\usepackage{listings}
\usepackage{color}
\definecolor{mygreen}{rgb}{0,0.6,0}
\definecolor{mygray}{rgb}{0.5,0.5,0.5}
\definecolor{mymauve}{rgb}{0.58,0,0.82}
\lstset{ %
  backgroundcolor=\color{white},   % choose the background color; you must add \usepackage{color} or \usepackage{xcolor}
  basicstyle=\footnotesize,        % the size of the fonts that are used for the code
  breakatwhitespace=false,         % sets if automatic breaks should only happen at whitespace
  breaklines=true,                 % sets automatic line breaking
  captionpos=b,                    % sets the caption-position to bottom
  commentstyle=\color{mygreen},    % comment style
  deletekeywords={...},            % if you want to delete keywords from the given language
  escapeinside={\%*}{*)},          % if you want to add LaTeX within your code
  extendedchars=true,              % lets you use non-ASCII characters; for 8-bits encodings only, does not work with UTF-8
  frame=single,                    % adds a frame around the code
  keepspaces=true,                 % keeps spaces in text, useful for keeping indentation of code (possibly needs columns=flexible)
  keywordstyle=\color{blue},       % keyword style
  language=Octave,                 % the language of the code
  morekeywords={*,...},            % if you want to add more keywords to the set
  numbers=left,                    % where to put the line-numbers; possible values are (none, left, right)
  numbersep=5pt,                   % how far the line-numbers are from the code
  numberstyle=\tiny\color{mygray}, % the style that is used for the line-numbers
  rulecolor=\color{black},         % if not set, the frame-color may be changed on line-breaks within not-black text (e.g. comments (green here))
  showspaces=false,                % show spaces everywhere adding particular underscores; it overrides 'showstringspaces'
  showstringspaces=false,          % underline spaces within strings only
  showtabs=false,                  % show tabs within strings adding particular underscores
  stepnumber=2,                    % the step between two line-numbers. If it's 1, each line will be numbered
  stringstyle=\color{mymauve},     % string literal style
  tabsize=2,                       % sets default tabsize to 2 spaces
  title=\lstname                   % show the filename of files included with \lstinputlisting; also try caption instead of title
}
\usepackage{caption}
\usepackage{url}

\begin{document}

\tableofcontents

\chapter{Introduction}

\section{GNU Radio}
GNU Radio\cite{bib:gnuradio} is a free \& open-source software development toolkit
that provides signal processing blocks to implement software radios.
It can be used with readily-available low-cost external RF hardware
to create software-defined radios, or without hardware in a simulation-like environment.
It is widely used in hobbyist, academic and commercial environments 
to support both wireless communications research and real-world radio systems.

A software radio is a radio system which performs the required signal processing 
in software instead of using dedicated integrated circuits in hardware.
The benefit is that since software can be easily replaced in the radio system,
the same hardware can be used to create many kinds of radios for many different transmission standards;
thus, one software radio can used for a variety of applications.

GNU Radio performs all the signal processing.
You can use it to write applications to receive data out of digital streams 
or to push data into digital streams, which is then transmitted using hardware.
GNU Radio has filters, channel codes, synchronisation elements, equalizers,
demodulators, vocoders, decoders, and many other elements 
(in the GNU Radio jargon, we call these elements blocks) which are typically found in radio systems.
More importantly, it includes a method of connecting these blocks 
and then manages how data is passed from one block to another.
Extending GNU Radio is also quite easy; if you find a specific block that is missing, you can quickly create and add it.

Since GNU Radio is software, it can only handle digital data.
Usually, complex baseband samples are the input data type for receivers and the output data type for transmitters.
Analog hardware is then used to shift the signal to the desired centre frequency.
That requirement aside, any data type can be passed from one block to another 
- be it bits, bytes, vectors, bursts or more complex data types.

GNU Radio applications are primarily written using the Python programming language, 
while the supplied, performance-critical signal processing path is implemented in C++
using processor floating point extensions, where available.
GNU Radio Companion(GRC) is a Simulink-like graphical tool to design signal processing flow graphs.

GNU Radio supports several radio front-ends, either natively or through additional out-of-tree modules.
We will be using Ettus Research USRP platform and RTL-SDR TV tuners.

\section{IQ Modulator Board}
The IQ modulator board features Linear Technology's LTC 5598\cite{bib:modchip}
High Linearity Direct Quadrature
Modulator, which works in the range from 5MHz to 1600MHz. 
It allows direct modulation of an RF signal using differential I and Q signals.
The I/Q baseband inputs consist of voltage-to-current converters that in turn drive double-balanced mixers.
The outputs of these mixers are summed and applied to a buffer,
which converts the differential mixer signals to a $50\Omega$ single-ended buffered RF output.

The carrier to the modulator is supplied from Linear Technology's LTC 6946-1\cite{bib:pllchip}, 
an ultra-low noise and spurious integer-N synthesizer with integrated VCO. 
This chip is a high performance, low noise phase-locked loop (PLL) with a fully integrated VCO,
including a reference divider, phase-frequency detector (PFD) with phase-lock indicator,
ultralow noise charge pump, integer feedback divider, and VCO output divider.
It's frequency range is from 373MHz to 3740MHz. 
With the combination of these chips the transmitter can communicate between 373MHz to 1600MHz.

The synthesizer can be programmed through SPI interface so as to select different frequencies of operation.
We use microchip's PIC18F4550 as the SPI controller and also for USB interfacing.
The modulator board is designed to work either using the USB power or from external power supply.


%%% References
\begin{thebibliography}{9}
 \bibitem{bib:gnuradio}
 \url{http://gnuradio.org}
 \bibitem{bib:modchip}
 \url{http://www.linear.com/product/LTC5598} 
 \bibitem{bib:pllchip}
 \url{http://www.linear.com/product/LTC6946} 
 
 \end{thebibliography}

\end{document}


